%% report.tex
%% Copyright 2023 Tom M. Ragonneau and Zaikun Zhang
\documentclass{article}
\usepackage[a4paper]{geometry}

% Manage culturally-determined typographical rules
\usepackage[british]{babel}

% Font and encoding
\usepackage{lmodern}
\usepackage[T1]{fontenc}

% Space between lines
\usepackage[nodisplayskipstretch]{setspace}
\singlespacing

% Subliminal refinements
\usepackage{microtype}
\usepackage[prevent-all]{widows-and-orphans}

% Modify the typesets of the `\maketitle' and `\thanks' commands
\usepackage{titling}
\setlength{\thanksmargin}{0pt}

% Mathematics
\usepackage{mathtools}
\usepackage{amssymb}

% Headings
\title{Revision no.\ 1 of GOMS-2023-0041}
\author{Tom M. Ragonneau \and Zaikun Zhang}
\date{\today}

\begin{document}

\maketitle

We would like to thank the reviewers for their comments and suggestions.
We have addressed the issues raised by the reviewers and have made the several changes to the manuscript.
The changes are highlighted in the revised manuscript using a blue color.

\section{Response to Reviewer no.\ 1}

\begin{enumerate}
    \item \textit{Verify the order of the citations. See, for instance, Page~2, Line~6 (should be~[1,3]) and Line~7 (should be~[5,11-13,23]).}\\
    We have corrected the order of the citations in the manuscript.
    \item \textit{Page~2, Line~29 : ``is impracticable'' instead of ``is impracticable in practice''.}\\
    The suggested change has been made.
    \item \textit{Page~6: Verify the statement of Lemma 3.1 by considering:}
    \begin{enumerate}
        \item \textit{The notations could be clarified after the lemma, such that~$x_j$ and~$\sum_{j \in \emptyset} x_j = 0$.}\\
        We chose to keep the notations employed in the Lemma when stating that an empty sum equals to zero.
        This is because we do not want to introduce new notations that are not used elsewhere in the manuscript to avoid confusion.
        \item \textit{The statement should be: ``For all~$x \in \mathbb{R}^n$ and~$m \in \{ n + 2, \dots, 2n + 1 \}$, the expression of~$L_i$, for~$i \in \{ 1, \dots, m \}$, is given by\dots''}\\
        The suggested change has been made.
        \item \textit{It would be more natural to state the expression of~$L_i$ for~$m - n + 1 \le i \le n + 1$ and next for~$n + 2 \le i \le m$.}\\
        The suggested change has been made.
    \end{enumerate}
    \item \textit{Page~8, Line~24: To avoid mentioning part of the proof, include the result about~$\Lambda_{\infty}$ as a particular case on the statement of Theorem~3.1.}\\
    The theorem has been modified to include the result about~$\Lambda_{\infty}$, and the sentence mentioning the proof has been modified to reflect this change.
    \item \textit{Page~9, Line~18: State Lemma~3.3 for~$\max_{\lVert x \rVert_p \le \Delta} \lVert x \rVert_q$, for a given~$\Delta > 0$, as it is invoked on the proof of Proposition~3.2 (Page~9, Line~44, with~$q = 2$).}\\
    For simplicity, we decided not to modify the Lemma.
    Instead, we explicited a change of variable in~(3.11), corresponding to the proof of Proposition~3.2.
    With this change of variable, the Lemma is now invoked directly.
    \item \textit{Page~12, Line~25: ``Frobenius'' instead of ``frobenius''.}
    The suggested change has been made.
    \item \textit{Include in the references:\\ \texttt{https://www.tandfonline.com/doi/full/10.1080/10556788.2015.1026968}\\ This article, published at Optimization Methods and Software, also uses the interpolation set as discussed here.}\\
    A citation to this article has been added to the manuscript (see the second paragraph of the introduction).
\end{enumerate}

\section{Response to Reviewer no.\ 2}

\begin{enumerate}
    \item \textit{Pg.~3, line~24. Please add a reference to the notion of ``well-poised sense''.}\\
    A reference to~\cite[\S~3.3]{Conn_Scheinberg_Vicente_2009} has been added.
    \item \textit{Pg.~4, line~10. ``Minimum Frobenius norm Lagrange polynomials''? This section deals with minimum Frobenius norm, correct?}\\
    All references to ``minimum-norm'' have been replaced by ``minimum Frobenius norm.''
    \item \textit{Pg.~6, Lemma~3.1. I think that there is some kind of mistake here or maybe the order could be improved. The second term is related to the interval~$2 \le i \le m - 1$, but the third one goes from~$n + 1$ to~$m$. Please verify this formula and change the third and fourth lines.}\\
    We have verified the formula and switched the third and fourth lines.
    \item \textit{Pg.~6, line~50. $L_j$ instead of~$L_i$.}
    The suggested change has been made.
    \item \textit{Pg.~6, line~51. Please add the extra information that this equation comes from the Taylor expansion of the quadratic, or something similar to this.}\\
    The suggested change has been made.
    \item \textit{Pg.~7, line~6. Please add ``On the other hand, by (3.4)''}\\
    The suggested change has been made.
    \item \textit{Pg.~8, line~26. I think it should be~$\le$ here, as in the second part of Theorem~3.1 you provide an upper bound for~$L_1$ and~$-L_1$. Also, the comment on~$2n - m + 2 \le n$ could be eliminated, as it was also part of the second inequality in the Theorem.}\\
    The equality is correct, but the proof has been improved to explain explicitely why.
    \item \textit{Pg.~8, line~45. I was unable to obtain the second inequality. Using H{\"{o}}lder inequality, I only obtained}
    \begin{equation*}
        \sum x_j \le (2n + 1 - m)^{\frac{p}{p - 1}} \lVert x \rVert_p
    \end{equation*}
    \textit{Note that the power is different from the manuscript. The same comment applies to line~52.}\\
    We verified thouroughly the proof of the proposition and found no mistake.
    Let~$z \in \mathbb{R}^n$ be defined by
    \begin{equation*}
        z_j = \begin{cases}
            0 & \text{if~$j \in \{ 1, \dots, m - n - 1 \}$,} \\
            1 & \text{otherwise.}
        \end{cases}
    \end{equation*}
    The H{\"{o}}lder inequality then provides
    \begin{equation*}
        \sum_{j = m - n}^n x_j = z^{\mathsf{T}} x \le \lVert z \rVert_{p / (p - 1)} \lVert x \rVert_p = (2n + 1 - m)^{\frac{p - 1}{p}} \lVert x \rVert_p.
    \end{equation*}
    \item \textit{Pg.~10, line~17. I did not understand the phrase ``This equality turns out to be true for a larger range of p.''. Why? Based on practical evidence? Please give references, if possible.}\\
    A consequence of Theorem~3.2 is that~$\Lambda_p = 1$ if~$p \le 2 \log n / \log (n / 2)$.
    The right-hand side of this inequality is greater than~$2$ for~$n \ge 2$.
    We modified the sentence in the manuscript to make this more explicit.
    \item \textit{Pg.~10, line~43. Please add ``well-poisedness in the minimum Frobenius norm sense, of an interpolation\dots''.}\\
    The suggested change has been made.
    \item \textit{Pg.~10, line~44. Add reference to the works of Powell.}\\
    The references to~\cite{Powell_2006,Powell_2009} have been added.
\end{enumerate}

\bibliographystyle{plain}
\bibliography{report}

\end{document}
