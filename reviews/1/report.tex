%% report.tex
%% Copyright 2023 Tom M. Ragonneau and Zaikun Zhang
\documentclass{article}
\usepackage[a4paper,margin=1in]{geometry}
\usepackage{enumitem}

% Manage culturally-determined typographical rules
\usepackage[british]{babel}

% Font and encoding
\usepackage{lmodern}
\usepackage[T1]{fontenc}

% Space between lines
\usepackage[nodisplayskipstretch]{setspace}
\singlespacing

% Subliminal refinements
\usepackage{microtype}
\usepackage[prevent-all]{widows-and-orphans}

% Modify the typesets of the `\maketitle' and `\thanks' commands
\usepackage{titling}
\setlength{\thanksmargin}{0pt}

% Mathematics
\usepackage{mathtools}
\usepackage{amssymb}

% Headings
\title{Revision of GOMS-2023-0041}
\author{Tom M. Ragonneau \and Zaikun Zhang}
\date{\today}

\begin{document}

\maketitle

We thank the reviewers for their helpful comments and suggestions.
We have addressed the issues raised by the reviewers.
The changes are highlighted in the revised manuscript in blue.

We provide below detailed responses to the comments of the reviewers.
We apologize if we switched the order of the reviewers, as it is unknown to us.
The equation and citation numbers in the responses correspond to those in the revised manuscript.

\section{Response to Reviewer 1}

\begin{enumerate}
    \item \textit{Verify the order of the citations. See, for instance, Page~2, Line~6 (should be~[1,3]) and Line~7 (should be~[5,11-13,23]).}\\
    We have corrected the order of the citations in the manuscript.
    \item \textit{Page~2, Line~29 : ``is impracticable'' instead of ``is impracticable in practice''.}\\
    The suggested change has been made.
    \item \textit{Page~6: Verify the statement of Lemma 3.1 by considering:}
    \begin{enumerate}
        \item \textit{The notations could be clarified after the lemma, such that~$x_j$ and~$\sum_{j \in \emptyset} x_j = 0$.}\\
        Thank you for the comment.
        In the original version of the manuscript, we wrote the following under equation~(3.4):\\[\baselineskip]
        ``\textit{Here,~$x_j$ denotes the~$j$th entry of~$x$ for each~$j \in \{ 1, 2, \dots, n \}$ and we define~$\sum_{j = m - n}^n x_j = 0$ in the formulation of~$L_1$ if~$m = 2n + 1$.}''\\[\baselineskip]
        We choose to keep this part unchanged in the revised version.
        We believe that the definition of~$x_j$ is clear.
        In addition, we did not mention~$\sum_{j \in \emptyset} x_j = 0$ because the manuscript does not use the notation of ``summation over sets.''
        Instead, we wrote that ``\textit{$\sum_{j = m - n}^n x_j = 0$ in the formulation of~$L_1$ if~$m = 2n + 1$}'' because this is the only case where the summation is empty (note that~$m \le 2n + 1$ in Lemma~3.1).
        \item \textit{The statement should be: ``For all~$x \in \mathbb{R}^n$ and~$m \in \{ n + 2, \dots, 2n + 1 \}$, the expression of~$L_i$, for~$i \in \{ 1, \dots, m \}$, is given by\dots''}\\
        The suggested change has been made.
        \item \textit{It would be more natural to state the expression of~$L_i$ for~$m - n + 1 \le i \le n + 1$ and next for~$n + 2 \le i \le m$.}\\
        We have exchanged the last two cases in equation~(3.4).
    \end{enumerate}
    \item \textit{Page~8, Line~24: To avoid mentioning part of the proof, include the result about~$\Lambda_{\infty}$ as a particular case on the statement of Theorem~3.1.}\\
    The theorem has been modified to include the result about~$\Lambda_{\infty}$.
    The proof is revised accordingly, and we have also adapted the first two sentences of Subsection~3.2.3 to reflect this change.
    \item \textit{Page~9, Line~18: State Lemma~3.3 for~$\max_{\lVert x \rVert_p \le \Delta} \lVert x \rVert_q$, for a given~$\Delta > 0$, as it is invoked on the proof of Proposition~3.2 (Page~9, Line~44, with~$q = 2$).}\\
    Thank you for the remark.
    We agree that our original version can be made clearer.
    For simplicity, we decided not to modify the lemma.
    Instead, we apply a change of variable in~(3.13), which enables us to invoke Lemma 3.3 directly.
    \item \textit{Page~12, Line~25: ``Frobenius'' instead of ``frobenius''.}\\
    The suggested change has been made.
    \item \textit{Include in the references:\\ \texttt{https://www.tandfonline.com/doi/full/10.1080/10556788.2015.1026968}\\ This article, published at Optimization Methods and Software, also uses the interpolation set as discussed here.}\\
    A citation of this article has been added to the manuscript in the second paragraph of the introduction.
\end{enumerate}

\section{Response to Reviewer 2}

\begin{enumerate}
    \item \textit{Pg.~3, line~24. Please add a reference to the notion of ``well-poised sense''.}\\
    A reference to~[5, \S~3.3] has been added to the penultimate paragraph of the introduction.
    \item \textit{Pg.~4, line~10. ``Minimum Frobenius norm Lagrange polynomials''? This section deals with minimum Frobenius norm, correct?}\\
    All the ``minimum-norm'' in the manuscript have been replaced with ``minimum Frobenius norm.''
    The Frobenius norm is the only norm considered for the Hessians of the models in this paper.
    \item \textit{Pg.~6, Lemma~3.1. I think that there is some kind of mistake here or maybe the order could be improved. The second term is related to the interval~$2 \le i \le m - 1$, but the third one goes from~$n + 1$ to~$m$. Please verify this formula and change the third and fourth lines.}\\
    Thank you for the comment.
    The original version of the formula is correct, but we agree that the order of the lines can be improved.
    We have switched the third and fourth lines in equation~(3.4) in the revised version.
    \item \textit{Pg.~6, line~50. $L_j$ instead of~$L_i$.}\\
    The suggested change has been made.
    \item \textit{Pg.~6, line~51. Please add the extra information that this equation comes from the Taylor expansion of the quadratic, or something similar to this.}\\
    We added ``Taylor expansions of the quadratic polynomial~$L$ around~$y^1$ yield'' in the first two lines of page~7.
    \item \textit{Pg.~7, line~6. Please add ``On the other hand, by (3.4)''}\\
    We write ``On the other hand, it is easy to check according to~(3.4) that all the entries of~$\nabla^2 L_i$ are zero except for the first~$m - n - 1$ diagonal entries'' before ending the proof of Lemma~3.1.
    \item \textit{Pg.~8, line~26. I think it should be~$\le$ here, as in the second part of Theorem~3.1 you provide an upper bound for~$L_1$ and~$-L_1$. Also, the comment on~$2n - m + 2 \le n$ could be eliminated, as it was also part of the second inequality in the Theorem.}\\
    The equality is correct, but we agree that some explanation is needed.
    Thank you for pointing this out.
    We have modified the proof of Theorem~3.1 to explain why this is an equality.
    See the last paragraph of the proof.
    \item \textit{Pg.~8, line~45. I was unable to obtain the second inequality. Using H{\"{o}}lder inequality, I only obtained}
    \begin{equation*}
        \sum x_j \le (2n + 1 - m)^{\frac{p}{p - 1}} \lVert x \rVert_p
    \end{equation*}
    \textit{Note that the power is different from the manuscript. The same comment applies to line~52.}\\
    Thank you for the comment, but we think there is no mistake here.
    Let~$z \in \mathbb{R}^n$ be defined by
    \begin{equation*}
        z_j = \begin{cases}
            0 & \text{if~$1 \le j \le m - n - 1$,} \\
            1 & \text{otherwise.}
        \end{cases}
    \end{equation*}
    The H{\"{o}}lder inequality then provides
    \begin{equation*}
        \sum_{j = m - n}^n x_j = z^{\mathsf{T}} x \le \lVert z \rVert_{p / (p - 1)} \lVert x \rVert_p = (2n + 1 - m)^{\frac{p - 1}{p}} \lVert x \rVert_p.
    \end{equation*}
    We note that the power of~$2n + 1 - m$ is~$(p - 1) / p$ even though it comes from the~$\ell_{p / (p - 1)}$-norm.
    \item \textit{Pg.~10, line~17. I did not understand the phrase ``This equality turns out to be true for a larger range of p.''. Why? Based on practical evidence? Please give references, if possible.}\\
    Thank you for the comment.
    We agree this needs clarification.
    In the new version, after the proof of Theorem~3.2, we write ``Theorem~3.2 shows that the equality holds for a larger range of~$p$, because the right-hand side in~(3.14) is greater than~$2$ when~$n > 2$.''
    \item \textit{Pg.~10, line~43. Please add ``well-poisedness in the minimum Frobenius norm sense, of an interpolation \dots''.}\\
    We have changed the first sentence of Section~4 to ``We have analyzed the well-poisedness (in the minimum Frobenius norm sense) of an interpolation \dots''.
    \item \textit{Pg.~10, line~44. Add reference to the works of Powell.}\\
    The references to~[19, 20] have been added.
    See the second line of the conclusion.
\end{enumerate}

\end{document}
