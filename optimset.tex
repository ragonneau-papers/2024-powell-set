%% Makefile
%% Copyright 2023 Tom M. Ragonneau and Zaikun Zhang
\documentclass[draft]{article}
\usepackage[utf8]{inputenc}
\usepackage[a4paper]{geometry}
\usepackage{microtype}

% Font and font encoding
\usepackage{lmodern}
\usepackage[T1]{fontenc}

% Line spacing
\usepackage{setspace}
\onehalfspacing

% Cross-referencing
\usepackage[pdfusetitle]{hyperref}
\usepackage{url}
\hypersetup{
    final,
    colorlinks=true,
    allcolors=black,
}
\newcommand{\email}[1]{\href{mailto:#1}{#1}}
\newcommand{\orcid}[1]{\href{https://orcid.org/#1}{#1}}

% Glossaries
\usepackage[acronym]{glossaries}
\glsdisablehyper
\newacronym{dfo}{DFO}{derivative-free optimization}

% Line numbering in draft mode
\usepackage{lineno}
\usepackage[dvipsnames]{xcolor}
\usepackage{ifdraft}
\renewcommand{\linenumberfont}{\normalfont\footnotesize\ttfamily\color{gray}}
\ifdraft{\linenumbers}{}

% To-do notes in draft mode
\usepackage[
    obeyDraft,
    colorinlistoftodos,
    color=OliveGreen!40,
    textsize=scriptsize,
]{todonotes}

% Modify the typesets of the `\maketitle' and `\thanks' commands
\usepackage{titling}
\setlength{\thanksmargin}{0pt}

% Exceptions for American English hyphenation patterns
\input{ushyphex}

% Document metadata
\title{An Optimal Initial Interpolation Set for Model-Based Derivative-Free Optimization Methods}
\author{
    Tom M. Ragonneau\thanks{
        Department of Applied Mathematics, The Hong Kong Polytechnic University, Hong Kong.\\
        Email: \email{tom.ragonneau@polyu.edu.hk}\\
        ORCID: \orcid{0000-0003-2717-2876}
    } \and
    Zaikun Zhang\thanks{
        Department of Applied Mathematics, The Hong Kong Polytechnic University, Hong Kong.\\
        Email: \email{zaikun.zhang@polyu.edu.hk}\\
        ORCID: \orcid{0000-0001-8934-8190}
    }
}
\date{\today}
\hypersetup{
    pdfsubject={},
    pdfkeywords={},
}

\begin{document}

\maketitle

\begin{abstract}
    To do.\todo{Do the abstract}
\end{abstract}

\section{Introduction}

It is challenging to design optimization methods that do not rely on classical or generalized derivative information of the problems they solve, namely \gls{dfo} methods.
However, these methods are necessary for solving problems whose derivatives are not available or too expensive to compute.
There are many examples of applications of this kind, such as hyperparameter tuning in machine learning~\cite{Ghanbari_Scheinberg_2017}, derivative-free reinforcement learning~\cite{Qian_Yu_2021}.
\todo{Add more examples}

\section{Description of the initial interpolation set}

To do.\todo{Adapted from~\cite{Powell_2006}}

\bibliographystyle{plain}
\bibliography{\jobname}

\listoftodos

\end{document}