%% Makefile
%% Copyright 2023 Tom M. Ragonneau and Zaikun Zhang
\documentclass[draft]{article}
\usepackage[utf8]{inputenc}
\usepackage[a4paper]{geometry}
\usepackage{microtype}

% Font and font encoding
\usepackage{lmodern}
\usepackage[T1]{fontenc}

% Line spacing
\usepackage[nodisplayskipstretch]{setspace}
\onehalfspacing

% Font Awesome 5
\usepackage{fontawesome5}
\usepackage[dvipsnames]{xcolor}
\definecolor{ORCIDGreen}{HTML}{A6CE39}

% Cross-referencing
\usepackage[
    pdfusetitle,
    hyperfootnotes=false,
]{hyperref}
\usepackage{url}
\hypersetup{
    final,
    colorlinks=true,
    linktoc=page,
    linkcolor=OliveGreen,
    anchorcolor=black,
    citecolor=MidnightBlue,
    filecolor=black,
    menucolor=black,
    runcolor=black,
    urlcolor=black,
}
\newcommand{\email}[2][inbox]{\faIcon{#1} \href{mailto:#2}{#2}}
\newcommand{\phone}[2][phone]{\faIcon{#1} \href{tel:#2}{#2}}
\newcommand{\orcid}[1]{\texorpdfstring{\hspace{1.5ex}\href{https://orcid.org/#1}{\textcolor{ORCIDGreen}{\faIcon{orcid}}}}{}}
\newcommand{\website}[2][globe]{\faIcon{#1} \href{https://#2/}{#2}}

% Glossaries
\usepackage[acronym]{glossaries}
\glsdisablehyper
\newacronym{dfo}{DFO}{derivative-free optimization}

% Mathematics
\usepackage{mathtools}
\usepackage[bb=dsfontserif]{mathalfa}
\numberwithin{equation}{section}

% Line numbering in draft mode
\usepackage{lineno}
\usepackage{ifdraft}
\renewcommand{\linenumberfont}{\normalfont\scriptsize\sffamily\color{gray}}
\ifdraft{\linenumbers}{}

% Show labels in draft mode
\ifdraft{}{\PassOptionsToPackage{final}{showlabels}}
\usepackage{showlabels}
\renewcommand{\showlabelfont}{\linespread{1}\scriptsize\sffamily\color{gray}}

% To-do notes in draft mode
\usepackage[
    marginface=\linespread{1}\scriptsize\sffamily\color{gray},
    author={},
]{fixme}

% Modify the typesets of the `\maketitle' and `\thanks' commands
\usepackage{titling}
\setlength{\thanksmargin}{0pt}

% Page headers and footers
\usepackage{fancyhdr}
\usepackage{lastpage}
\fancypagestyle{plain}{
    \fancyhf{}
    \renewcommand{\headrulewidth}{0pt}
    \fancyfoot[C]{Page \thepage\ of \pageref*{LastPage}}
}
\pagestyle{plain}

% Exceptions for American English hyphenation patterns
\input{ushyphex}

% Mathematical environments (theorems, proofs, ...)
\usepackage{amsthm}
\usepackage{mdframed}
\mdfdefinestyle{defaultstyle}{
    bottomline=false,
    innertopmargin=0pt,
    innerrightmargin=0pt,
    linecolor=gray,
    linewidth=1pt,
    rightline=false,
    rightmargin=0pt,
    topline=false,
}
\theoremstyle{definition}
\newtheorem{definition}{Definition}[section]
\surroundwithmdframed[style=defaultstyle]{definition}

% Control sequences, macros, and definitions
\usepackage{etoolbox}
\usepackage{xspace}
\DeclareMathOperator{\bigo}{\mathcal{O}}
\DeclareMathOperator{\card}{card}
\newcommand{\abs}[2][]{#1\lvert#2#1\rvert}
\newcommand{\fset}{\Omega}
\newcommand{\lagp}[1][]{L\ifblank{#1}{}{_{#1}}}
\newcommand{\norm}[2][]{#1\lVert#2#1\rVert}
\newcommand{\obj}{f}
\newcommand{\objm}[1][]{\hat{\obj}\ifblank{#1}{}{^{#1}}}
\newcommand{\qpoly}{\mathcal{Q}_n}
\newcommand{\R}{\mathbb{R}}
\newcommand{\set}[2][]{#1\{#2#1\}}
\newcommand{\solvername}[1]{\textsc{#1}\xspace}
\newcommand{\xpt}[1][]{\mathcal{Y}\ifblank{#1}{}{^{#1}}}

% Document metadata
\title{An Optimal Initial Interpolation Set for Model-Based Derivative-Free Optimization Methods}
\author{
    Tom M. Ragonneau\thanks{
        Department of Applied Mathematics, The Hong Kong Polytechnic University, Hong Kong\\
        \email{tom.ragonneau@polyu.edu.hk} \quad \website[globe-asia]{www.tomragonneau.com}
    }\orcid{0000-0003-2717-2876} \and
    Zaikun Zhang\thanks{
        Department of Applied Mathematics, The Hong Kong Polytechnic University, Hong Kong\\
        \email{zaikun.zhang@polyu.edu.hk} \quad \website[globe-asia]{www.zhangzk.net}
    }\orcid{0000-0001-8934-8190}
}
\date{\today}
\hypersetup{
    pdfsubject={},
    pdfkeywords={},
}

\begin{document}

\maketitle

\begin{abstract}
    \fxnote*{Write down the abstract}{To do.}
\end{abstract}

\section{Introduction}

It is challenging to develop \gls{dfo} methods, i.e., optimization methods that do not rely on derivatives of the problems they solve.
However, such methods are required to solve the optimization problem
\begin{equation}
    \label{eq:nlp}
    \begin{aligned}
        \min_{x \in \R^n}   & \quad \obj(x)\\
        \text{s.t.}         & \quad x \in \fset,
    \end{aligned}
\end{equation}
when classical or generalized derivatives of the objective function~$\obj : \R^n \to \R$ (and possibly the constraint functions underlying the feasible~$\fset \subseteq \R^n$) are unavailable or too expensive to evaluate.
There are many examples of applications of this kind, such as machine learning~\cite{Ghanbari_Scheinberg_2017}, reinforcement learning~\cite{Qian_Yu_2021}, particle physics~\cite{Eldred_Etal_2022}, and aircraft engineering~\cite{Gazaix_Etal_2019}.

Broadly speaking, \gls{dfo} methods can be divided into two categories: direct-search and model-based methods~\cite{Conn_Scheinberg_Vicente_2009}.
Direct-search methods~\cite{Kolda_Lewis_Torczon_2003} generate sample points in the search space around the current iterate, and select the next iterate among these points based on simple comparisons.
Examples of famous direct-search methods include the Hooke-Jeeves method~\cite{Hooke_Jeeves_1961}, the Nelder-Mead method~\cite{Nelder_Mead_1965}, the \solvername{gps} method~\cite{Booker_Etal_1999}, the \solvername{mads} method~\cite{Audet_Dennis_2006}, and \solvername{bfo}~\cite{Porcelli_Toint_2017,Porcelli_Toint_2022}.
Meanwhile, \fxnote*{Find a citation that introduces the concept of model-based methods}{model-based methods} approximate the objective and constraint functions by simple models around the current iterate, and select the next iterate by examining these approximations.
Since these approximations are local, the methods are embedded in globalization strategies to ensure convergence.
The trust-region framework~\cite{Conn_Gould_Toint_2000} is a popular choice for this purpose, and is the one under consideration in this paper.
Examples of well-known trust-region methods include the method of Conn and Toint~\cite{Conn_Toint_1996}, \solvername{mnh}~\cite{Wild_2008}, Powell's \gls{dfo} methods~\cite{Powell_1994,Powell_2002,Powell_2006,Powell_2009}, and \solvername{cobyqa}~\cite{Ragonneau_2022,Ragonneau_Zhang_2023}.
% There also exist hybrid methods that combine the two approaches, such as implicit filtering~\cite{Kelley_2011}.

Model-based \gls{dfo} methods and in particular trust-region \gls{dfo} methods often employ polynomial interpolation schemes to approximate the objective and constraint functions.
% In particular, this is the case for the Powell's \gls{dfo} methods and \solvername{cobyqa}.
These methods maintain an interpolation set~$\xpt[k] \subseteq \R^n$, where~$k$ denotes the iteration number, and build a model~$\objm[k]$ of~$\obj$ according to the interpolation conditions
\begin{equation}
    \label{eq:intp-cond}
    \objm[k](y) = \obj(y), \quad y \in \xpt[k].
\end{equation}
% Note that the methods must ensure that the interpolation conditions~\eqref{eq:intp-cond} do not contradict each others.
We assume in what follows that the models~$\objm[k]$ are quadratic, and we denote by~$\qpoly$ the space of quadratic polynomials.
Such models are often employed in practice, for example by \solvername{uobyqa}~\cite{Powell_2002}, \solvername{newuoa}~\cite{Powell_2006}, \solvername{bobyqa}~\cite{Powell_2009}, \solvername{lincoa}, \solvername{mnh}, and \solvername{cobyqa}.
Clearly, if the interpolation conditions~\eqref{eq:intp-cond} do not contradict each others and
\begin{equation*}
    \card\xpt[k] = \dim\qpoly = \frac{1}{2} (n + 1) (n + 2),
\end{equation*}
then~$\objm[k]$ is uniquely defined by~\eqref{eq:intp-cond}.
This scheme, referred to as \emph{fully-determined interpolation}, is the one employed by \solvername{uobyqa}.
However, the first iteration of this method requires~$\bigo(n^2)$ function evaluations to construct the initial model~$\objm[0]$.
This is impracticable in practice unless~$n$ is small.
Therefore, modern interpolation-based \gls{dfo} methods employ fewer interpolation points, giving rise to so-called \emph{underdetermined interpolation}.
To uniquely define~$\objm[k]$ in this case, it is frequently set to a solution to a variational problem of the form
\begin{equation}
    \label{eq:intp-var}
    \begin{aligned}
        \min_{Q \in \qpoly} & \quad \mathcal{F}^k(Q)\\
        \text{s.t.}         & \quad Q(y) = \obj(y), \quad y \in \xpt[k],
    \end{aligned}
\end{equation}
where~$\mathcal{F}^k$ is a functional that promotes a desired regularity of~$\objm[k]$.
For example, \solvername{mnh} employs
\begin{equation}
    \label{eq:mnh}
    \mathcal{F}^k(Q) = \norm[\big]{\nabla^2 Q}_{\mathsf{F}}^2,
\end{equation}
where~$\norm{\cdot}_{\mathsf{F}}$ denotes the Frobenius norm, and \solvername{newuoa}, \solvername{bobyqa}, \solvername{lincoa}, and \solvername{cobyqa} employ
\begin{equation}
    \label{eq:df-psb}
    \mathcal{F}^k(Q) = \norm[\big]{\nabla^2 Q - \nabla^2 \objm[k - 1]}_{\mathsf{F}}^2,
\end{equation}
with~$\objm[-1] = 0$.
The latter functional is inspired by least-change property of quasi-Newton updates~\cite{Dennis_Schnabel_1979}.
The variational problem defined by~\eqref{eq:intp-var} and~\eqref{eq:df-psb} is known as the \emph{derivative-free symmetric Broyden update} (see~\cite{Powell_2013},~\cite[\S~3.6]{Zhang_2012}, and~\cite[\S~2.4.2]{Ragonneau_2022}).
The two underdetermined interpolation schemes assume that
\begin{equation*}
    n + 1 \le \card(\xpt[k]) \le \frac{1}{2} (n + 1) (n + 2).
\end{equation*}
Otherwise,~$\objm[k]$ could not be uniquely defined (or even not defined at all).
In practice, methods that employ these schemes often enforce~$\card(\xpt[k]) \ge n + 2$ as all the models~$\objm[k]$ would be linear if~$\card(\xpt[k]) = n + 1$.
% Quadratic models of~$\con$ can be built in a similar way.
Finally, if the constraint functions underlying the feasible set~$\fset$ in~\eqref{eq:nlp} are nonlinear, model-based \gls{dfo} methods usually approximates them using similar techniques.

Normally, interpolation-based \gls{dfo} methods do not build the interpolation point~$\xpt[k]$ \emph{ab initio} at each iteration.
Instead, they build an initial interpolation set~$\xpt[0]$ and then  update~$\xpt[k]$ to build~$\xpt[k + 1]$.
For example, only one point differs between~$\xpt[k]$ and~$\xpt[k + 1]$ in Powell's \gls{dfo} methods and \solvername{cobyqa} during usual iterations.
This mechanism prevents the methods from performing unnecessary function evaluations, which are often considered as the principal measure of complexity of \gls{dfo} methods.
However, it is clear that the initial interpolation set~$\xpt[0]$ must be carefully chosen, as it will impact the optimization method for many iterations.
In the unconstrained case, i.e., when~$\fset = \R^n$ in~\eqref{eq:nlp}, Powell proposed an initial interpolation set~$\xpt[0]$ that is employed by many interpolation-based \gls{dfo} methods, including \solvername{uobyqa}, \solvername{newuoa}, \solvername{bobyqa}, \solvername{lincoa}, and \solvername{cobyqa}.
This paper shows that this interpolation set is optimal in a well-poisedness sense detailed in Section~\ref{sec:well-poisedness}.

This paper is organized as follows.\fxnote{Write down the organization of the paper}

\section{Well-poisedness of interpolation sets}
\label{sec:well-poisedness}

In this section, we denote by
\begin{equation*}
    \xpt = \set{y^1, y^2, \dots, y^m} \subseteq \R^n,
\end{equation*}
a set, and we interpolate the objective function~$\obj$ by a quadratic polynomial~$\objm$ on~$\xpt$, i.e.,
\begin{equation}
    \label{eq:intp-cond-gen}
    \objm(y^i) = \obj(y^i), \quad i \in \set{1, 2, \dots, m}.
\end{equation}
More precisely, we let~$\objm$ solve~\eqref{eq:intp-var}, where~$\mathcal{F}^k$ is defined by~\eqref{eq:mnh}.
Note that if~$\xpt$ is seen as an initial interpolation set~$\xpt[0]$, then~$\objm$ denotes the initial model~$\objm[0]$, and the functional~$\mathcal{F}^k = \mathcal{F}^0$ is equivalent to the one defined by~\eqref{eq:df-psb}.
In other words,~$\objm$ is also obtained by the derivative-free symmetric Broyden update.
The following definition guarantees that the interpolation problem is always well-posed.

\begin{definition}[Poisedness]
    The set~$\xpt$ is \emph{poised} in the minimum Frobenius norm sense if the solution to problem~\eqref{eq:intp-var} exists and is unique for any real-valued function~$\obj$.
\end{definition}

As we mentioned earlier, a necessary condition for~$\xpt$ to be poised is
\begin{equation*}
    n + 1 \le m \le \frac{1}{2} (n + 1) (n + 2).
\end{equation*}
Moreover, observe that the KKT system for the variational problem~\eqref{eq:intp-var} is linear with respect to the coefficient of~$\objm$.
Hence, the poisedness of~$\xpt$ is related to the nonsingularity of the coefficient matrix of this KKT system.
Intuitively, the interpolation set~$\xpt$ can be said well-poised if this matrix is well-conditioned.
To formalize this intuition, we first introduce the notion of minimum Frobenius norm Lagrange polynomials.

\subsection{Minimum Frobenius norm Lagrange polynomials}

We assume that the interpolation set~$\xpt$ is poised.
To formally define a measure of well-poisedness of~$\xpt$, we first need to extend the classical definition of the Lagrange polynomials as follows.

\begin{definition}[Minimum Frobenius norm Lagrange polynomials]
    \label{def:min-norm-lagp}
    For each~$i \in \set{1, 2, \dots, m}$, the~$i$th \emph{minimum Frobenius norm Lagrange polynomial}~$\lagp[i]$ for the interpolation problem~\eqref{eq:intp-cond-gen} is the unique quadratic polynomial solving
    \begin{equation*}
        \begin{aligned}
            \min_{Q \in \qpoly} & \quad \norm{\nabla^2 Q}_{\mathsf{F}}\\
            \text{s.t.}         & \quad Q(y^j) = \delta_{i, j}, \quad j \in \set{1, 2, \dots, m},
        \end{aligned}
    \end{equation*}
    where~$\delta_{i, j}$ denotes the Kronecker delta.
\end{definition}

Remark that if~$m = (n + 1) (n + 2) / 2$, then Definition~\ref{def:min-norm-lagp} reduces to the classical definition of the Lagrange polynomials.
Moveover, it can be shown that~$\objm$ is given by
\begin{equation*}
    \objm(x) = \sum_{i = 1}^m \obj(y^i) \lagp[i](x),
\end{equation*}
for any~$x \in \R^n$.

\subsection{Well-poisedness in the minimum Frobenius norm sense}

We are now equiped to define the notion of~$\Lambda$-poisedness in the minimum Frobenius norm sense.

\begin{definition}[$\Lambda$-poisedness in the minimum Frobenius norm sense~{\cite[Def.~5.6]{Conn_Scheinberg_Vicente_2009}}]
    The poised interpolation set~$\xpt$ is said to be~\emph{$\Lambda$-poised in the minimum Frobenius norm sense} in a compact set~$\mathcal{C} \subseteq \R^n$, for some~$\Lambda > 0$, if
    \begin{equation*}
        \Lambda \ge \max_{1 \le i \le m} \max_{x \in \mathcal{C}} \abs{\lagp[i](x)}.
    \end{equation*}
\end{definition}

If~$\xpt$ is~$\Lambda_0$-poised in the minimum Frobenius norm sense in~$\mathcal{C}$, it is obviously~$\Lambda$-poised in the minimum Frobenius norm sense in~$\mathcal{C}$ for all~$\Lambda \ge \Lambda_0$.
Moreover, one can show that~$\xpt$ is~$\Lambda$-poised if and only if the condition number of the coefficient matrix of the KKT system of~\eqref{eq:intp-var} is bounded by some terms proportional to~$\Lambda$~\cite[Thm.~5.8]{Conn_Scheinberg_Vicente_2009}.
Therefore, the notion of~$\Lambda$-poisedness formalizes the intuitive notion of well-poisedness mentioned above.

\section{Optimality of Powell's initial interpolation set}

\section{Conclusion}

% \section{Optimality of an initial interpolation set}

% In what follows, we present an initial interpolation set and its optimality in a sense describe in Subsection~\ref{subsec:well-poisedness}.

% \subsection{Description of the initial interpolation sets}

% The initial interpolation set we consider in this paper is defined as follows.
% Let~$\Delta > 0$ be fixed and for~$i \in \set{1, 2, \dots, 2n + 1}$, let~$y^i \in \R^n$ be
% \begin{equation*}
%     y^i =
%     \begin{cases}
%         0,                      & \text{if} ~ i = 1,\\
%         \Delta e_{i - 1},       & \text{if} ~ 2 \le i \le n + 1,\\
%         -\Delta e_{i - n - 1},  & \text{otherwise},
%     \end{cases}
% \end{equation*}
% where~$e_i \in \R^n$ denotes the $i$th standard coordinate vector.
% We then define the initial interpolation sets~$\xpt[0]_m \subseteq \R^n$ for each~$m \in \set{n + 2, n + 3, \dots, 2n + 1}$ by
% \begin{equation*}
%     \xpt[0]_m = \set{y^1, y^2, \dots, y^m}.
% \end{equation*}
% Under these notations, the interpolation conditions~\eqref{eq:intp-cond} provide
% \begin{equation}
%     \label{eq:intp-cond-0}
%     \objm[0](y^i) = \obj(y^i), \quad i \in \set{1, 2, \dots, m}.
% \end{equation}
% % In the unconstrained case, Powell's \gls{dfo} methods and \solvername{cobyqa} use the initial interpolation set~$\xpt[0] = \xpt[0]_{2n + 1}$, i.e., they employ the value~$m = 2n + 1$ by default.
% % This is natural, as the corresponding initial interpolation set is appealing.
% % This paper shows that the value~$m = 2n + 1$ is indeed optimal in a sense described in the following section.

% \subsection{Well-poisedness of interpolation sets}
% \label{subsec:well-poisedness}

% In this part, we fix an~$m \in \set{n + 2, n + 3, \dots, 2n + 1}$.
% We will study properties of the initial interpolation set~$\xpt[0]_m$ in $\mathcal{B}_p(\Delta)$, the~$\ell_p$-norm ball of radius~$\Delta$, for~$p \in [1, \infty]$.
% Note that we allow~$p = \infty$.
% It is natural to focus on this region.
% Indeed, the initial interpolation set is contained in~$\mathcal{B}_p(\Delta)$ and more importantly, this is the region of interest when employing the initial interpolation set~$\xpt[0]_m$ in a trust-region framework.\fxnote{Introduce the trust-region framework before.}

% We first introduce the concept of minimum Frobenius norm Lagrange polynomials associated with the interpolation conditions~\eqref{eq:intp-cond-0}.

% \begin{definition}[Minimum Frobenius norm Lagrange polynomials]
%     For each~$i \in \set{1, 2, \dots, m}$, the~$i$th minimum Frobenius norm Lagrange polynomial~$\lagp[i]$ for the interpolation problem~\eqref{eq:intp-cond-0} is the unique quadratic polynomial that solves
%     \begin{equation*}
%         \begin{aligned}
%             \min_{Q \in \qpoly} & \quad \norm{\nabla^2 Q}_{\mathsf{F}}\\
%             \text{s.t.}         & \quad Q(y^j) = \delta_{i, j},\\
%         \end{aligned}
%     \end{equation*}
%     where~$\delta_{i, j}$ denotes the Kronecker delta.
% \end{definition}

% We now present the notion of~$\Lambda$-poisedness in the minimum Frobenius norm sense.

% \begin{definition}[$\Lambda$-poisedness in the minimum Frobenius norm sense~{\cite[Def.~5.6]{Conn_Scheinberg_Vicente_2009}}]
%     An interpolation set~$\xpt[0]_m$ is said to be~$\Lambda$-poised in the minimum Frobenius norm sense in~$\mathcal{B}_p(\Delta)$, for some~$\Lambda > 0$, if
%     \begin{equation*}
%         \Lambda \ge \max_{1 \le i \le m} \max_{\norm{x}_p \le \Delta} \abs{\lagp[i](x)},
%     \end{equation*}
%     where~$\lagp[i]$ denotes the~$i$th minimum Frobenius norm Lagrange polynomials.
% \end{definition}

% If~$\xpt[0]_m$ is~$\Lambda_0$-poised in the minimum Frobenius norm sense, it is obviously~$\Lambda$-poised in the minimum Frobenius norm sense for all~$\Lambda \ge \Lambda_0$.
% Moreover, one can show that the set~$\xpt[0]_m$ is~$\Lambda$-poised if and only if the condition number of the coefficient matrix of the KKT system of~\eqref{eq:intp-var} (with~$\mathcal{F}^k$ defined by either~\eqref{eq:mnh} or~\eqref{eq:df-psb}) is bounded by some terms proportional to~$\Lambda$~\cite[Thm.~5.8]{Conn_Scheinberg_Vicente_2009}.


% \subsection{Optimality in the well-poisedness sense}

% These methods are trust-region methods and~$\Delta$ represent the initial trust-region radius.

\bibliographystyle{amsplain}
\bibliography{\jobname}

\listoffixmes

\end{document}