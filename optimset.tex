%% Makefile
%% Copyright 2023 Tom M. Ragonneau and Zaikun Zhang
\documentclass[draft]{article}
\usepackage[utf8]{inputenc}
\usepackage[a4paper]{geometry}
\usepackage{microtype}

% Font and font encoding
\usepackage{lmodern}
\usepackage[T1]{fontenc}

% Line spacing
\usepackage{setspace}
\onehalfspacing

% Cross-referencing
\usepackage[pdfusetitle]{hyperref}
\usepackage{url}
\hypersetup{
    final,
    colorlinks=true,
    allcolors=black,
}
\newcommand{\email}[1]{\href{mailto:#1}{#1}}
\newcommand{\orcid}[1]{\href{https://orcid.org/#1}{#1}}

% Glossaries
\usepackage[acronym]{glossaries}
\glsdisablehyper
\newacronym{dfo}{DFO}{derivative-free optimization}

% Line numbering in draft mode
\usepackage{lineno}
\usepackage[dvipsnames]{xcolor}
\usepackage{ifdraft}
\renewcommand{\linenumberfont}{\normalfont\footnotesize\ttfamily\color{gray}}
\ifdraft{\linenumbers}{}

% To-do notes in draft mode
\usepackage{fixme}

% Modify the typesets of the `\maketitle' and `\thanks' commands
\usepackage{titling}
\setlength{\thanksmargin}{0pt}

% Exceptions for American English hyphenation patterns
\input{ushyphex}

% Document metadata
\title{An Optimal Initial Interpolation Set for Model-Based Derivative-Free Optimization Methods}
\author{
    Tom M. Ragonneau\thanks{
        Department of Applied Mathematics, The Hong Kong Polytechnic University, Hong Kong\\
        Email: \email{tom.ragonneau@polyu.edu.hk}\\
        ORCID: \orcid{0000-0003-2717-2876}
    } \and
    Zaikun Zhang\thanks{
        Department of Applied Mathematics, The Hong Kong Polytechnic University, Hong Kong\\
        Email: \email{zaikun.zhang@polyu.edu.hk}\\
        ORCID: \orcid{0000-0001-8934-8190}
    }
}
\date{\today}
\hypersetup{
    pdfsubject={},
    pdfkeywords={},
}

\begin{document}

\maketitle

\begin{abstract}
    \fxnote*{Do the abstract}{To do.}
\end{abstract}

\section{Introduction}

It is challenging to develop \gls{dfo} methods, i.e., optimization methods that do not rely on derivatives of the problems they solve.
However, such methods are required to solve optimization problems when classical or generalized derivatives of the objective and constraint functions are unavailable or too expensive to evaluate.
There are many examples of applications of this kind, such as machine learning~\cite{Ghanbari_Scheinberg_2017}, reinforcement learning~\cite{Qian_Yu_2021}, particle physics~\cite{Eldred_Etal_2022}, and aircraft engineering~\cite{Gazaix_Etal_2019}.

Broadly speaking, \gls{dfo} methods can be divided into two categories: direct-search~\cite{Kolda_Lewis_Torczon_2003} and model-based methods.
Direct-search methods sample the objective function around the current iterate, and select the next iterate among the sample points based on simple comparisons.

\section{Description of the initial interpolation set}

\fxnote*{Adapted from~\cite{Powell_2006}}{To do.}

\bibliographystyle{amsplain}
\bibliography{\jobname}

\listoffixmes

\end{document}